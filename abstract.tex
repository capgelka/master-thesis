% \begin{eabstract}
% ...
% \end{eabstract}

\begin{abstract}
В настоящее время фаззинг-тестирование -- активно развивающийся метод для поиска ошибок в программном обеспечении \cite{DBLP:journals/corr/abs-1808-09700}. Тем не менее, у него имеется ряд недостатков, в частности достижение некоторых фрагментов кода может быть затруднено. Одним из способов решения этой проблемы может быть передача фаззеру информации о влиянии входных данных на выполнение условных переходов. Для получения этой информации предлагается воспользоваться методами динамического анализа. В данной работе проводится обзор подходящих средств динамического анализа, а также предлагаются два подхода к решению поставленной задачи. Первый подход основан на динамическом символьном выполнении, для него предлагается метод с программной реализацией на основе инструмента Angr, протестированный на искуственных примерах. Второй подход основывается на динамическом анализе помеченных данных, для него предлагается метод на основе инструмента gentrace из проекта Moflow, а также соответствующие доработки данного инструмента. Полученное решение протестировано на программах из набора LAVA и проектах с открытым исходным кодом.


% Работа содержит \total{page} страницу, \total{truechapters} главы, \total{figurecnt} рисунков, \total{tablecnt} таблицу, \total{bibcnt} источников, \total{appendixchapters} приложения.
\end{abstract}

\selectlanguage{english}
\begin{abstract}
Fuzz-testing is an actively developed and used method for software errors search \cite{DBLP:journals/corr/abs-1808-09700}. However there are some disadvantages, for example it may be difficult to reach some code with it. This can be resolved with by giving information about bytes which influenced on branch execution. We made an overview of dynamic analysis frameworks and propose two approaches to get the information. The first approach based on dynamic symbolic execution, we also propose a method based on the Angr framework as its implementation. It was tested on synthetic examples. The second approach is based on dynamic taint analysis. We propose improvements for the moflow gentrace tool to be used as the implementation of the method based on the second approach. This solution was tested on LAVA test set and open source programms.
\end{abstract}
\selectlanguage{russian}
