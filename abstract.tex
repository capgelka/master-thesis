% \begin{eabstract}
% ...
% \end{eabstract}

\begin{abstract}
В настоящее время фаззинг-тестирование -- активно развивающийся метод для поиска ошибок в программном обеспечении \cite{DBLP:journals/corr/abs-1808-09700}. Тем не менее, у него имеется ряд недостатков, в частности достижение некоторых фрагментов кода может быть затруднено. Одним из способов решения этой проблемы может быть передача фаззеру информации о входных данные влияющих на выполнение условных переходов. Для получения необходимой информации предлагается воспользоваться методами динамического анализа. В данной работе проводится обзор подходящих средств динамического анализа. Также предлагаются 2 конкретных метода, на основе динамического символьного выполнения и на основе анализа помеченных данных. Для метода на основе символьного выполнения приводится программная реализация на основе фреймворка Angr, протестированная на искуственных примерах. Для метода на основе анализа помеченных данных предлагаются доработки инструмента gentrace из проекта Moflow. Полученное решение тестируется на программах из набора LAVA.


% Работа содержит \total{page} страницу, \total{truechapters} главы, \total{figurecnt} рисунков, \total{tablecnt} таблицу, \total{bibcnt} источников, \total{appendixchapters} приложения.
\end{abstract}

\selectlanguage{english}
\begin{abstract}
Here translation should be
\end{abstract}
\selectlanguage{russian}
