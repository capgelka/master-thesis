% \begin{eabstract}
% ...
% \end{eabstract}

\begin{abstract}
В настоящее время фаззинг-тестирование -- активно развивающийся метод для поиска ошибок в программном обеспечении \cite{DBLP:journals/corr/abs-1808-09700}. Тем не менее, у него имеется ряд недостатков, в частности достижение некоторых фрагментов кода может быть затруднено. Для решения этой проблемы фаззеру необходимо знать какие входные данные влияют на выполнение условных переходов. Получить необходимую информацию можно при помощи динамического анализа. В данной работе приводится обзор подходов, позволяющих решать эту задачу, а также подходящих инструментов динамического анализа. Кроме того, были предложены и разработаны методы на основе динамического анализа помеченных данных и динамического символьного выполнения. 

Работа содержит \total{page} страницу, \total{truechapters} \total{totalchapters} главы, \total{figurecnt} рисунков, \total{tablecnt} таблицу, \total{bibcnt} источников, \total{appendixchapters} приложения.
\end{abstract}
