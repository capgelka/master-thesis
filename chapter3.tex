\chapter{Выбор базы для дальнейшей работы}

Поскольку подход с использованием динамического символьного выполнения оказался немасштабируем, в условиях использования одного из рассмотренных интрументов, встает вопрос о выборе технологии динамического анализа помеченных данных.

В данной главе описывается метод сравнения инструментов динамического анализа помеченных данных, и применяется для cравнения следующих из них.

\begin{itemize}
    \item Triton
    \item Taintgrind
    \item libdft64
    \item moflow gentrace
\end{itemize}

\section{Подход к сравнению}

Поскольку разные инструменты поддерживают различную гранулярность меток, и максимальное количество источников для одного адреса имеет смысл проводить сравнение тех характеристик, которые присутствуют у всех инструментов -- количества помеченных условных переходов, а также времени выполнения и затраченной памяти.

Для сравнения времени выполнения и используемой памяти можно использовать утилиту \texttt{time} из проекта \texttt{GNU} \cite{TIME}. Однако единственым инструментом, позволяющим получить информацию о помеченных условных переходов является \texttt{Triton}, во всем остальных случаях необходимо вносить изменения в исходный код инструмента. 

Чтобы избежать дублирования кода была разработана библиотека, предоставляющая интерфейс для записи трасы с информацией о помеченных условных переходах и серилизацией полученной информации в json формат. Для moflow, taintgrind и libdft64 были разработы патчи для записи трассы при попощи указанной библиотеки. Для Triton была написана утилита, собирающая интересующую информацию с использованием библиотеки и Triton API.

Поскольку в libdft регистр флагов не представлен, была использована следующая эвристика -- условный переход считался помеченным, если перед ним находилась инструкция \texttt{cmp} у который хотя бы один из аргументов помечен. Для большей точности стоило бы также рассмотреть аналогичный подход с инструкцией \texttt{test}. В силу устройства современных компиляторов, такой подход является достаточно точным.

Для возможности эффективной отладки и быстрого воспроизведения результатов экспериментов были также разработаны скрипты на языке \texttt{Python} для параллельного запуска инструмента на различных тестовых наборах, и для генерирования таблиц с результатами соответствующего запуска.

В качестве тестового набора для сравнения использовались приложения из набора \texttt{LAVA} \cite{LAVA} и утилита \texttt{cmark}, используемая для преобразования markdown в html.

% Мы просто описываем метрики, которые сняли тут, не углублясь в детали про Rust и сложности линковки?

\section{Результаты сравнения}

% \begin{table}[H]
%     \caption{Время работы в секундах для различных реализаций} \label{tab:compare}
%     \scalebox{0.8}{
%     \begin{tabular}[]{@{}lllllllll@{}}
%     \toprule
%     & vanilla & vector & bitset6000 & roaring & bitset64 tree & bitset256
%     tree & bitset512 tree & interval set \tabularnewline
%     \midrule
%     % \endhead
%     cmark & 3.5s & 4.6s & 3.7s & 4.2s & 3.8s & 3.7s & 3.7s & 3.7s \tabularnewline
%     file & 20.8s & 47s & 60s & 70s & 45.5s & 46s & 47s & 46s \tabularnewline
%     libjpeg & 14.5s & 1762s & 49s & 378s & 307s & 108s & 80s & 165s \tabularnewline
%     libyaml & 16.5s & 22.5s & 25s & 26s & 23.5s & 23.5s & 23.5s & 24s \tabularnewline
%     \bottomrule
% \end{tabular}}
% \end{table}

% \begin{table}[H]
% \caption{Taintgrind} \label{tab:c1}
% \scalebox{0.3}{
% \begin{tabular}[]{@{}llllllll@{}}
% \toprule
% & Test name &
% Suite &
% Binary &
% Size, KB &
% Trace size &
% Taken tainted branches &
% Tainted branches &
% Time, &
% min:sec:ms &
% Memory, MB \tabularnewline
% \midrule
% Grep & LAVA & 603 & 3227 & 29 & 62 & 0:04.57 & 155 \tabularnewline
% Libyaml & LAVA & 427 & 3717 & 101 & 494 & 3:42.04 & 182 \tabularnewline
% Pcre2grep & LAVA & 918 & 2689 & 42 & 111 & 0:10.30~ & 153 \tabularnewline
% jq & LAVA & 2700 & 2421 & 27 & 78 & 0:31.64 & 124 \tabularnewline
% Jpeg & LAVA & 1013 & 2282 & 169 & 305 & 31:34.51 & 1443 \tabularnewline
% cmark & Real-world & 322 & 2497 & 111 & 266 & 0:24.28~ &
% 152\tabularnewline
% toy & LAVA & 17 & 1725 & 14 & 38 & 0:04.32 & 152 \tabularnewline
% jumper & synthetic & 17 & 1575 & 7 & 12 & 0:04.17~ & 152 \tabularnewline
% \bottomrule
% \end{tabular}}
% \end{table}

\begin{longtable}[]{@{}llllllll@{}}
\caption{Taintgrind} \label{tab:c1}\\
\toprule
\begin{minipage}[b]{0.12\columnwidth}\raggedright\strut
Test name\strut
\end{minipage} & \begin{minipage}[b]{0.12\columnwidth}\raggedright\strut
Suite\strut
\end{minipage} & \begin{minipage}[b]{0.12\columnwidth}\raggedright\strut
Binary

Size, KB\strut
\end{minipage} & \begin{minipage}[b]{0.12\columnwidth}\raggedright\strut
Trace size\strut
\end{minipage} & \begin{minipage}[b]{0.12\columnwidth}\raggedright\strut
Taken tainted branches\strut
\end{minipage} & \begin{minipage}[b]{0.12\columnwidth}\raggedright\strut
Tainted branches\strut
\end{minipage} & \begin{minipage}[b]{0.12\columnwidth}\raggedright\strut
Time,

min:sec:ms\strut
\end{minipage} & \begin{minipage}[b]{0.12\columnwidth}\raggedright\strut
Memory, MB\strut
\end{minipage}\tabularnewline
\midrule
\endhead
Grep & LAVA & 603 & 3227 & 29 & 62 & 0:04.57 & 155\tabularnewline
Libyaml & LAVA & 427 & 3717 & 101 & 494 & 3:42.04 & 182\tabularnewline
Pcre2grep & LAVA & 918 & 2689 & 42 & 111 & 0:10.30~ & 153\tabularnewline
jq & LAVA & 2700 & 2421 & 27 & 78 & 0:31.64 & 124\tabularnewline
Jpeg & LAVA & 1013 & 2282 & 169 & 305 & 31:34.51 & 1443\tabularnewline
cmark & Real-world & 322 & 2497 & 111 & 266 & 0:24.28~ &
152\tabularnewline
toy & LAVA & 17 & 1725 & 14 & 38 & 0:04.32 & 152\tabularnewline
jumper & synthetic & 17 & 1575 & 7 & 12 & 0:04.17~ & 152\tabularnewline
\bottomrule
\end{longtable}

Тут просто таблицы с комментариями.

Непонятно как прокомментировать разницу в количестве инструкций, особенно когда в качестве DBI один и тот же инструмент - Pin.

\section{Выбор moflow}

В результате качественного сравнения инструментов, проделанного в обзоре и количественного в предыдущем пункте, было принято решение выбрать в качестве базы для дальнейшей работы \texttt{moflow gentrace}, как быстрый инструмент, поддерживающий байтовую гранулярность меток, отслеживание помеченности регистра флагов и распространение пометок для всех x86-64 инструкций.

% libdft не работал и нет поддержки регистра флагов, triton медленный, taintgrind с ``плохим кодом''?