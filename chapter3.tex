\chapter{Выбор базы для дальнейшей работы}

Поскольку подход с использованием динамического символьного выполнения оказался немасштабируем, в условиях использования одного из рассмотренных инструментов, встает вопрос о выборе технологии динамического анализа помеченных данных.

В данной главе описывается метод сравнения инструментов динамического анализа помеченных данных, и применяется для сравнения следующих из них.

\begin{itemize}
    \item Triton
    \item Taintgrind
    \item libdft64
    \item moflow gentrace
\end{itemize}

\section{Подход к сравнению}

Поскольку разные инструменты поддерживают различную гранулярность меток, и максимальное количество источников для одного адреса имеет смысл проводить сравнение тех характеристик, которые присутствуют у всех инструментов -- количества помеченных условных переходов, а также времени выполнения и затраченной памяти.

Для сравнения времени выполнения и используемой памяти можно использовать утилиту \texttt{time} из проекта \texttt{GNU} \cite{TIME}. Однако единственным инструментом, позволяющим получить информацию о помеченных условных переходов является \texttt{Triton}, во всем остальных случаях необходимо вносить изменения в исходный код инструмента. 

Чтобы избежать дублирования кода была разработана библиотека, предоставляющая интерфейс для записи трассы с информацией о помеченных условных переходах и cериaлизацией полученной информации в json формат. Для moflow, taintgrind и libdft64 были разработаны патчи для записи трассы при помощи указанной библиотеки. Для Triton была написана утилита, собирающая интересующую информацию с использованием библиотеки и Triton API.

Поскольку в libdft регистр флагов не представлен, была использована следующая эвристика --- условный переход считался помеченным, если перед ним находилась инструкция \texttt{cmp} у который хотя бы один из аргументов помечен. Для большей точности стоило бы также рассмотреть аналогичный подход с инструкцией \texttt{test}. В силу устройства современных компиляторов, такой подход является достаточно точным.

Для возможности эффективной отладки и быстрого воспроизведения результатов экспериментов были также разработаны скрипты на языке \texttt{Python} для параллельного запуска инструмента на различных тестовых наборах, и для генерирования таблиц с результатами соответствующего запуска.

В качестве тестового набора для сравнения использовались приложения из набора \texttt{LAVA} \cite{LAVA} и утилита \texttt{cmark}, используемая для преобразования markdown в html. И ранее упомянутая утилита  jumper, исходный код которой приведен в листинге ~\ref{lst:jumper}.

% Мы просто описываем метрики, которые сняли тут, не углублясь в детали про Rust и сложности линковки?

\section{Результаты сравнения}


\begin{longtable}[]{@{}llllll@{}}
\caption{Результаты для Triton} \label{tab:triton}\\
\toprule
\begin{minipage}[b]{0.12\columnwidth}\raggedright\strut
\strut
\end{minipage} & \begin{minipage}[b]{0.16\columnwidth}\raggedright\strut
Количество условных переходов\strut
\end{minipage} & \begin{minipage}[b]{0.16\columnwidth}\raggedright\strut
Выполненных условных переходов\strut
\end{minipage} & \begin{minipage}[b]{0.16\columnwidth}\raggedright\strut
Помеченных условных переходов\strut
\end{minipage} & \begin{minipage}[b]{0.16\columnwidth}\raggedright\strut
Время работы,

min:sec:ms\strut
\end{minipage} & \begin{minipage}[b]{0.16\columnwidth}\raggedright\strut
Используемая память, MB\strut
\end{minipage}\tabularnewline
\midrule
\endhead
grep & 47093 & 24464 & 8226 & 1:24.47 & 1975\tabularnewline
libyaml & 2037853 & 1177067 & 335785 & 39:55.45 &
47915\tabularnewline
pcre2grep & 51544 & 28501 & 11061 & 1:27.81 &
1627\tabularnewline
jpeg & 283884 & 225553 & 223853 & 34:25.76 &
32902\tabularnewline
cmark & 75766 & 31311 & 15677 & 1:30.86 &
2120\tabularnewline
toy & 14670 & 8573 & 30 & 0:18.45 & 383\tabularnewline
jumper & 13879 & 8245 & 16 & 0:16.02 & 359\tabularnewline
\bottomrule
\end{longtable}


\begin{longtable}[]{@{}llllll@{}}
\caption{Результаты для Taintgrind} \label{tab:taintgrind}\\
\toprule
\begin{minipage}[b]{0.12\columnwidth}\raggedright\strut
\strut
\end{minipage} & \begin{minipage}[b]{0.16\columnwidth}\raggedright\strut
Количество условных переходов\strut
\end{minipage} & \begin{minipage}[b]{0.16\columnwidth}\raggedright\strut
Выполненных условных переходов\strut
\end{minipage} & \begin{minipage}[b]{0.16\columnwidth}\raggedright\strut
Помеченных условных переходов\strut
\end{minipage} & \begin{minipage}[b]{0.16\columnwidth}\raggedright\strut
Время работы,

min:sec:ms\strut
\end{minipage} & \begin{minipage}[b]{0.16\columnwidth}\raggedright\strut
Используемая память, MB\strut
\end{minipage}\tabularnewline
\midrule
\endhead
Grep &  3227 & 29 & 62 & 0:04.57 & 155\tabularnewline
Libyaml &  3717 & 101 & 494 & 3:42.04 & 182\tabularnewline
Pcre2grep &  2689 & 42 & 111 & 0:10.30~ & 153\tabularnewline
jq & 2421 & 27 & 78 & 0:31.64 & 124\tabularnewline
Jpeg & 2282 & 169 & 305 & 31:34.51 & 1443\tabularnewline
cmark & 2497 & 111 & 266 & 0:24.28~ &
152\tabularnewline
toy & 1725 & 14 & 38 & 0:04.32 & 152\tabularnewline
jumper & 1575 & 7 & 12 & 0:04.17~ & 152\tabularnewline
\bottomrule
\end{longtable}



\begin{longtable}[]{@{}llllll@{}}
\caption{Результаты для libdft64} \label{tab:libdft}\\
\toprule
\begin{minipage}[b]{0.12\columnwidth}\raggedright\strut
\strut
\end{minipage} & \begin{minipage}[b]{0.16\columnwidth}\raggedright\strut
Количество условных переходов\strut
\end{minipage} & \begin{minipage}[b]{0.16\columnwidth}\raggedright\strut
Выполненных условных переходов\strut
\end{minipage} & \begin{minipage}[b]{0.16\columnwidth}\raggedright\strut
Помеченных условных переходов\strut
\end{minipage} & \begin{minipage}[b]{0.16\columnwidth}\raggedright\strut
Время работы,

min:sec:ms\strut
\end{minipage} & \begin{minipage}[b]{0.16\columnwidth}\raggedright\strut
Используемая память, MB\strut
\end{minipage}\tabularnewline
\midrule
\endhead
grep & 4167 & 375 & 178 & 0:05.94 & 496\tabularnewline
libyaml & 4206 & 395 & 520 & 0:30.13 & 231\tabularnewline
pcre2grep & 3346 & 328 & 872 & 0:04.75 & 230\tabularnewline
jpeg & 2266 & 174 & 291 & 0:03.15 & 230\tabularnewline
cmark & 2997 & 297 & 340 & 0:04.54 & 231\tabularnewline
toy  & 2131 & 193 & 182 & 0:02.84 & 226\tabularnewline
jumper & 1920 & 175 & 181 & 0:02.55 & 226\tabularnewline
\bottomrule
\end{longtable}


\begin{longtable}[]{@{}llllll@{}}
\caption{Результаты для moflow gentrace} \label{tab:moflow}\\
\toprule
\begin{minipage}[b]{0.12\columnwidth}\raggedright\strut
\strut
\end{minipage} & \begin{minipage}[b]{0.16\columnwidth}\raggedright\strut
Количество условных переходов\strut
\end{minipage} & \begin{minipage}[b]{0.16\columnwidth}\raggedright\strut
Выполненных условных переходов\strut
\end{minipage} & \begin{minipage}[b]{0.16\columnwidth}\raggedright\strut
Помеченных условных переходов\strut
\end{minipage} & \begin{minipage}[b]{0.16\columnwidth}\raggedright\strut
Время работы,

min:sec:ms\strut
\end{minipage} & \begin{minipage}[b]{0.16\columnwidth}\raggedright\strut
Используемая память, MB\strut
\end{minipage}\tabularnewline
\midrule
\endhead
grep & 975 & 287 & 24 & 0:02.17 & 44\tabularnewline
libyaml & 3415652 & 1152692 & 926490 & 0:47.53 &
518\tabularnewline
pcre2grep & 53655 & 16224 & 12366 & 0:13.29 &
492\tabularnewline
jpeg & 486365 & 218363 & 460995 & 1:29.91 &
494\tabularnewline
cmark & 101065 & 21379 & 37880 & 0:12.15 &
489\tabularnewline
toy & 2166 & 574 & 1723 & 0:07.25 & 338\tabularnewline
jumper & 796 & 210 & 23 & 0:03.72 & 42\tabularnewline
\bottomrule
\end{longtable}


% Тут просто таблицы с комментариями.

% Непонятно как прокомментировать разницу в количестве инструкций, особенно когда в качестве DBI один и тот же инструмент - Pin.

\section{Выбор moflow}

Разница в размере трасс объясняется разным подходом к инструментации. Можно видеть, что только два инструмента работают за приемлемое время - это \texttt{moflow gentrace} и \texttt{libdft64}. Однако в \texttt{libdft64} отсутствует поддержка SSE инструкций, что является критичным требованием, поскольку такие инструкции присутствуют практически во всех достаточно больших приложениях, собранных современным компилятором.\footnote{В частности в cmark и примерах из набора LAVA}

Таким образом, в результате качественного сравнения инструментов, проделанного в обзоре и количественного в предыдущем пункте, было принято решение выбрать в качестве базы для дальнейшей работы \texttt{moflow gentrace}, как быстрый инструмент, поддерживающий байтовую гранулярность меток, отслеживание помеченности регистра флагов и распространение пометок для всех x86-64 инструкций.

% libdft не работал и нет поддержки регистра флагов, triton медленный, taintgrind с ``плохим кодом''?