\bigskip
\chapter*{Заключение}

Было проведено исследование методов динамического анализа и выявлены два различных метода для определения входных данных влияющих на выполнение условных переходов.

\begin{itemize}
    \item Метод на основе динамического символьного выполнения.
    \item Метод на основе динамического анализа помеченных данных.
\end{itemize}

Для метода на основе динамического символьного выполнения была выполнена программная реализация на фреймоворке \texttt{Angr} и протестирована на искусственных примерах. Недостатком данной реализации является низкая производительность и невозможность мастшабировать решение на программы большего размера.
Метод на основе динамического анализа помеченных данных был программно реализован посредством доработок инструмента \texttt{Moflow gentrace}, данное решение обладает достаточной производительностью и может использоваться для анализа используемых в индустрии приложений. Тестирование проводилось на на наборе \texttt{LAVA}.

% Таким образом поставленная задача в рамках работы решена.

В качестве дальнейших направлений развития работы можно назвать следующие:

\begin{itemize}
    \item Исследование возможностей оптимального использование фаззером полученной информации.
    \item Доработка модулей \texttt{Angr} с целью использования конкретного выполнения, исследование возможностей масштабирование средств динамического символьного выполнения на программы большего размера.
    \item Исследование возможностей улучшения точности инструмента анализа помеченных данных.
\end{itemize}


% В процессе решения поставленной задачи 
% \begin{itemize}

% \item Был проведен краткий обзор истории и современного состояния стандарт SMT-LIB\cite{smtlib} для работы с SMT солверами.
% \item Были исследованы методы транслирования промежуточных языков в формулы  формулы для SMT солверов в соответствии со стандартом SMTLIB2.
% \item Был проведен анализ существующих средств символьного выполнения, применимых для задач отладки программ/анализа двоичного кода.
% \item Был проведен анализ возможностей расширения gdb для интеграции со средствами символьного выполнения.
% \item Был проведен анализ возможностей расширения radare2 для интеграции со средствами символьного выполнения.
% \item Была разработана библиотека на языке python для трансляции бинарного кода в формат smt-lib2.0.


% \end{itemize}

% Была разработана python библиотека,
% позволяющая в простейших случаях осуществлять трансляцию бинарного кода кода в smt формулы, с использованием radare2 и вспомогательных библиотек для него.


\bigskip

% \bibliographystyle{abstract}
\bibliographystyle{ieeetr}
\bibliography{master-thesis}
