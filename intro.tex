\chapter{Введение}


В последние годны наблюдается заметный рост количества уязвимостей в программном обеспечении. Так, согласно статистике \cite{CVEstats}, в $2016$ году было обнаружено $6447$ уязвимостей, в $2017$ --  $14714$, a в $2018$ -- $16555$. Это связано как с объемом и сложностью, разрабатываемого программного обеспечения, так и с развитием техник тестирования безопасности.

\begin{figure}[h]
    \center{
        \includegraphics[scale=0.5]{img/cve_stats.png}
    }
    \caption{Cтатистика опубликованных уязвимостей за последние 20 лет}
    \label{fig:image}
\end{figure}

Одним из популярных подходов к автоматизации поиска уязвимостей является фаззинг-тестирование. Это
техника тестирования программного обеспечения, заключающая в автоматической генерации входных данных. Целью фаззинга является нахождение таких входных данных, которые вызовут аварийное завершение программы.

Для достижения своей цели фаззеру необходимо генерировать входные данные, позволяющие пройти по разным путям выполнения. Один из популярных подходов, впервые примененный в AFL \cite{AFL} заключается в мутировании изначальных входных данных с целью увеличения покрытия.

Несмотря на высокую эффективность такого подхода он не лишен недостатков. Так например для кода
\begin{lstlisting}[environoment=C_LANG, label=example1]
uint32_t buf[10];
read(10, &buf, 40);
buf[0] &= 500;

if (buf[0] == 100) {
    //branch 1
    ...
} else {
    //branch 2
    ...
}
\end{lstlisting}

для того, чтобы найти входные данные для прохождения по ветке $1$ необходимо фактически перебрать $2^32$ значений. Для решения подобных проблем в \cite{DRILLER} предлагается подход, основанный на динамическом символьном выполнении. В случае если фаззер не может сгенерировать входные данные для прохождения по некоторой ветки в течение некоторого ограниченного времени, запускается модуль символьного выполнения, который составляет и решает предикат пути. Затем фаззер продолжает свою работу.
Подход из \cite{DRILLER} реализован в инструменте \emph{DRILLER}, распространяемом под открытой лицензией.

Комбинирование фаззинга и символьного выполнения тоже имеет свои сложности:
\begin{itemize}
    \item Генерация формул для каждой инструкции достаточно времязатратная операция.
    \item Задача поиска решений символьных уравнений является \textbf{NP полной}, не смотря на наличие множества эвристик невозможно гарантировать решение предиката пути за разумное время.
\end{itemize}

В данной работе предлагается иной способ улучшения эффективности открытия фаззером новых путей. Так в \ref{example1} фаззер не знает от каких входных данных зависит условный переход и может мутировать все байты входного файлв для открытия ветки $1$, однако в действительности значение \textbf{buff[0]} определяется только одним байтом, и только его и имеет смысл мутировать.
Так если у фаззера будет возможность для каждого условного перехода узнать от каких входных данных зависит его выполнение, это позволит существенно сократить количество итераций, необходимых для открытия пути.

Предметом работы является исследование и разработка методов, позволяющих предоставить фаззеру искомую информацию.



\chapter{Обзор}

% \chapter{Методы реализации динамического анализа}

\section{Динамический Анализ}

Динамический анализ - это анализ, заключающийся в непосредственном выполнении кода. Однако, просто запустить программу может быть недостаточно. Существует несколько способов получения дополнительной информации во время выполнения:

\begin{itemize}
\item {\em Исполнение кода в виртуальном окружении}. При данном подходе программа запускается внутри некоторого программного эмулятора. Например qemu \cite{QEMU}.
%\item $\sigma$ is a {\em symbolic store} that associates program variables with expressions over \mynote{[D] $\alpha_i$ also concrete?} concrete and symbolic values $\alpha_i$.

\item {\em Статическая инструментация}.
Статическая инструментация бывает двух видов.
    \begin{itemize}
        \item {\em Статическая инструментация исходного кода}. В случае, если имеется доступ к исходному коду, можно просто внести изменения в текстовые файлы с кодом. Добавление отладочной печати может быть примером статической инструментации исходного кода. Подобный вид инструментации также поддерживается непосредственно компилятором. Так GCC имеет опцию  \textit{-finstrument-functions}
        \item {\em Статическая инструментация бинарного кода (SBI)}. В случае отсутствия исходного кода, изменениям может быть подвержен сам исполняемый файл на диске. Тривиальным примером такой инструментации может быть например замена условных переходов на \textint{nop} инструкции. 
    \end{itemize}

\item {\em Динамическая бинарная инструментация}. Данный вид инструментации позволяет вносить изменения в программу непосредственно в процессе её выполнения. Подробное описание возможных подходов к реализации динамической инструментации можно найти в \cite{PBA}.
\end{itemize}

В данной работе проводилось исследование инструментов, основанных на использовании эмуляторов и динамической бинарной инструментации.


\section{Динамическая бинарная инструментация}

Основная идея динамической инструментации заключается в том, что инструментирующая программа контролирует все выполняемые инструкции. Классическая реализация работает примерно следующим образом. Перед непосредственным запуском программы происходин настройка модуля динамической инструментации и устанавливаются функции обратного вызова для различных видов событий, затем код инструментируемой программой считывается по базовым блокам и выполняется.

Рассмотрим несколько популярных DBI инструментов.

\subsection{QDBI}

В \cite{QDBI} \emph{QDBI} - расшифровфывается как \emph{QuarkslaB Dynamic binary Instrumentation}, авторы позиционируют его как кросплатформенный и кроссархитектурный фреймворк динамической бинарной инструментации. В планах разработчиков поддержка операционных систем Linux, macOS, Android, iOS и Windows, а также архитектур  x86, x86-64, ARM и AArch64.

Однако, в настоящий момент полноценно поддерживаются только Linux, macOS, Windows на x86-64 архитектуре, при этом поддержки SIMD инструкций нет.

\emph{QDBI} 



\subsection{Valgrind}
...

\subsection{Pin}
...

\section{Анализ помеченных данных}

Динамический анализ помеченных данных (Dynamic Taint Analysis, DTA), также известный как динамический анализ потока данных (Dynamic Flow tracking, DFT)(DFT) - это техника анализа програм, позволяющая определить какие состояния программы зависят от входных данных.
% Существует также статический анализ потока данных

Примером классической задачи, решаемой при помощи анализа помеченных данных, может служить задача определения достигают ли данные из недоверенного источника "Опасных" функций. Многие уязвимости в программном обеспечении обусловлены недостаточным контролем над входными данным. Применение анализа потока данных позволяет детектировать подобные проблемы.

Динамический анализ помеченных данных делится на 3 фазы

    \begin{itemize}
        \item {\em Определение источников помеченных данных}. На данном этапе определяется каким данные должны быть помечены. Обычно метками снабжаются данные, получаемые из недоверенного
        источника. В зависимости от типа приложения, это могут быть данные полученные по сети, из файла, или потока стандартного ввода.
        \item {\em Распространение пометок (Taint propogation)}. Для отслеживания потока данных, 
        для каждой инструкции манипулирующей данными необходимо написать инструментирующий код для манипуляции метками. Так, например инструкция \textint{mov eax, ebx} перезаписывает метку для регистра eax меткой регистра \textint{ebx}. Это фаза является самой сложной, поскольку оставляет много открытых вопросов. К примеру
        \begin{itemize}
            \item Следует ли отслеживать помеченность побайтово или побитого? Если \textint{eax} помечен, то после команды \textint{or eax, 0x746567bc} контролируются уже не все биты. Однако, в большинстве случаев отслеживание каждого бита может быть слишком дорогой операцией.
            \item Следует ли помечать адрес памяти, на который указывает помеченная переменная?
            \item Если условный переход зависит от помеченных данных, следует ли считать что последующие инструкции тоже от них зависят?
            \item Как хранить информацию о помеченных адресах в памяти?
            \item Cледует ли различать пометки, полученные из разных источников?
        \end{itemize}
        \item {\em Применение политик безопасности}. Фаза, на которой используются результаты анализа. Происходящее на этом этапе зависит от изначальных целей анализа. Типичным примером может быть отслеживание попадания помеченных данных в аргументы некоторых заранее выделенных функций, или факта помеченности счетчика инструкций.
    \end{itemize}


% \section{методы реализации технологии анализа помеченных данных}
Рассмотрим несколько подходов к реализации анализа помеченных данных.

% \subsection{Символьное выполнение}

\subsection{Множество помеченных адресов}

\subsection{Хэш таблица с побайтовыми метками}

\section{Символьное выполнение}


\section{Обзор технологий динамического анализа}

\section{Triton}

\section{angr}

\section{manticore}

\section{taintgrind}

\section{libdft}

\section{moflow}



\chapter{Сравнение инструментов для динамического анализа}

\section{Библиотека для снятия и анализа трас}

\section{Результаты сравнение}

\section{Выводы}




\chapter{Методы определения входных данных влияющих на условные переходы}

% Для начала расмотрим самый простой способ

\section{Использование символьного выполнение}

% Символьное выполнение


\section{Использование меток помеченных данных}

\section{Комбинированный подход}

Оба предыдыщих подхода имеют недостатки. Построение символьных формул для всех инструкций может быть достаточно ресурсоемкой задачей. Время работы \em{Triton} может быть тому примером. В случае же, если используется онлайн символьное выполнение - проблема стоит еще острее, так \em{angr} вообще оказывается не очень применим на программах размером больше чем задания для CTF соревнований.
\\
С другой стороны, многие технологии анализа помеченных данных не поддерживают гранулярность на уровне отдельных байт, и возможность отследить от каких именно входных байт зависит некоторый адрес или регистр отсутствует. Даже если есть возможность отследить метки на каждый байт, существуют примеры когда этого недостаточно. так в \cite{Cavallaro07anti-taint-analysis:practical} приводится следующий пример, где между x и y есть взаимно-однозначное соответствие, которое не отслеживается динамическим анализом помеченных данных.
\\

\begin{lstlisting}[environoment=C_LANG]
char y[256], x[256];
...
int n = read(network, y, sizeof(y));
for (int i=0; i < n; i++) {
    switch (y[i]) {
        case 0: x[i] = (char)13; break;
        case 1: x[i] = (char)14; break;
        ...
        case 255: x[i] = (char)12; break;
        default: break;
    }
}
\end{lstlisting}

\chapter{Прототипы решающие задачу}

\section{Решение на основе Angr}

\section{Решение на основе Moflow}

\chapter{Заключение}
% \subsection 




 % Для решения этой проблемы может использоваться динамическое символьное выполнение, например Driller для фаззера afl \cite{DRILLER}. Для улучшения работы фаззера