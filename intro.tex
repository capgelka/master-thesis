\section{Введение.}

В настоящее время фаззинг-тестирование -- активно развивающийся метод для поиска ошибок в программном обеспечении \cite{DBLP:journals/corr/abs-1808-09700}. Тем не менее, у него имеется ряд недостатков, в частности достижение некоторых фрагментов кода может быть затруднено. Для решения этой проблемы фаззеру необходимо знать какие входные данные влияют на выполнение условных переходов. Получить необходимую информацию можно при помощи динамического анализа. В данной работе приводится обзор подходов инструментов, позволяющих решать эту задачу. А также предлагаются несколько конкретных методов с программной реализацией прототипа.


\section{Обзор}

Динамический анализ - это анализ, заключающийся в непосредственном выполнении. С точки зрения реализации можно выделить несколько способов осуществления динамического анализа.

\begin{itemize}
\item {\em Исполнение кода в виртуальном окружении}. При данном подходе программа запускается внутри некоторого программного эмулятора. Например qemu \cite{QEMU}
%\item $\sigma$ is a {\em symbolic store} that associates program variables with expressions over \mynote{[D] $\alpha_i$ also concrete?} concrete and symbolic values $\alpha_i$.

\item {\em Динамическая инструментация}.

\item {\em Статическая инструментация}.

\end{itemize}


 % Для решения этой проблемы может использоваться динамическое символьное выполнение, например Driller для фаззера afl \cite{DRILLER}. Для улучшения работы фаззера