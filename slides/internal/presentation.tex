\documentclass[10pt]{beamer}

\usepackage[utf8]{inputenc}
\usepackage[english,russian]{babel}

\usetheme[
  sectionpage=simple,
  numbering=fraction
]{metropolis}
\usepackage{appendixnumberbeamer}

\usepackage{booktabs}
\usepackage[scale=2]{ccicons}

\usepackage{pgfplots}
\usepgfplotslibrary{dateplot}

\usepackage{xspace}
\newcommand{\themename}{\textbf{\textsc{metropolis}}\xspace}

\usepackage{todonotes}
\usepackage{ifthen}%
\providecommand\enabletodos{true}%
\ifthenelse{ \equal{\enabletodos}{true} }{%
  \presetkeys{todonotes}{inline}{}%
}{%
  \presetkeys{todonotes}{disable}{}%
}%

\setbeamertemplate{caption}{\raggedright\insertcaption\par}

\AtBeginSection[]{}

\setbeamertemplate{section in toc}{%
  \alert{$\bullet$}~\inserttocsection}
\setbeamercolor{subsection in toc}{bg=white,fg=structure}
\setbeamertemplate{subsection in toc}{%
  \hspace{1.2em}{\alert{\rule[0.3ex]{3pt}{3pt}}~\inserttocsubsection\par}}

\usepackage{hyperref}
\hypersetup{unicode=true}

\usepackage{tikz}
\usepackage{forest}
\usetikzlibrary{arrows,positioning}
\tikzset{
    % Define standard arrow tip
    >=latex,
    % Define arrow style
    ptr/.style={->, thick},
}
\usepackage{drawstack}
\usepackage{adjustbox}
\protected\def\psverb#1{\def\innerpsverb##1#1{\texttt{##1}}\innerpsverb}
\usepackage{listings}

\usepackage{makecell}

\usepackage[noline, noend]{algorithm2e} % For algorithms
\SetAlFnt{\footnotesize}

\usepackage{float}

\usepackage{changepage}

\lstdefinestyle{C}{
  language=C,
  % numbers=left,
  stepnumber=1,
  % numbersep=10pt,
  % tabsize=4,
  showspaces=false,
  showstringspaces=false
}
\lstset{basicstyle=\tiny,style=C}

\title{Результаты работы за 1 квартал 2019}
\author{Леонид Дьячков}
\institute{ИСП РАН}
\date{4 Апрeля 2019}
\titlegraphic{\hfill\includegraphics[height=0.5cm]{logo_isp_ru.png}}

\definecolor{Blue}{HTML}{2375a8}
\setbeamercolor{frametitle}{bg=Blue}
\setbeamercolor{palette primary}{bg=Blue}
\begin{document}

\maketitle

% {%
% \setbeamertemplate{frame footer}{
% Падарян В.А., Соловьев М.А., Кононов А.И. "Моделирование операционной семантики
% машинных инструкций." Труды Института системного программирования РАН, том 19,
% стр. 165186. 2010.
% }
\begin{frame}{Глобальные задачи}
    Разработка инструмента, позволяющего определять какие байты во входных файлах влияют на условные переходы в программе. 
    \begin{itemize}
      \item Изучение инструментов динамического анализа помеченных данных и динамического символьного выполнения
      \item Разработка инструментов для тестирования инструментов и оценки качества их работы
      \item Доработка выбранного инструмента.
    \end{itemize}
\end{frame}

\begin{frame}{Решаемые подзадачи | OSS Fuzz}
    \textbf{Задача} Для сравнения инструментов использовать существующию инфраструктуру Google OSS-fuzz \\
    \textbf{Результаты}
    \begin{itemize}
      \item Изучена инфраструктура google os fuzz.
      \item Получено представление о работе afl и libfuzz (в меньшей степени) c точки зрения пользователя
      \item На jenkins заведен job, запускающий afl фаззер при помощи oss fuzz
      \item Понимание как адаптировать имеющуюся инфраструктуру для оценки инструментов taint анализа не получено
    \end{itemize}
\end{frame}
%}%

\begin{frame}{Решаемые подзадачи | Разработка библиотеки для снятия и анализа трасс}
    \textbf{Задача.} Для сравнения инструментов необходимо разработать вспомогательную библиотеку, решающую следующие задачи:
    \begin{itemize}
      \item Сбор информации об условных переходах (адрес, опкод, был ли совершен переход, является ли инструкция помеченной)
      \item Подсчет различных метрик (длина трассы, количество уникальных прыжков, количество помеченных прыжков)
      \item Возможность интеграции в проекты на языках С и python
    \end{itemize}
  \end{frame}

\begin{frame}{Решаемые подзадачи | Разработка библиотеки для снятия и анализа трасс}
    \textbf{Результаты:}
    \begin{itemize}
      \item Разработана и покрыта тестами библиотека \emph{insrumentation-lib} на языке программирование Rust, предоставляющая интерфейсы для сбора трассы и подсчета метрик, а также поддерживающая сериализацию и десериализацию трассы.
      \item Реализовна сборка в виде динамической библиотеки, статической библиотеки и в виде библиотеки для языков python2 и python3 
      \item По техническим (PinCRT) причинам оказалось невозможно использовать библиотеку как предполагалось в проектах, используюших pin3.
    \end{itemize}
\end{frame}

\begin{frame}{Решаемые подзадачи | Изучение инструментов }
    \textbf{Задача.} Изучить реализацию и методы работы существующих инструментов динамического анализа, интегрировать в них разработанную библиотеку
    \begin{itemize}
        \item triton
        \item libdft
        \item taintgrind
        \item moflow
    \end{itemize}
\end{frame}

\begin{frame}{Решаемые подзадачи | Изучение инструментов }
    \textbf{Результаты.} Изучить реализацию и методы работы существующих инструментов динамического анализа, интегрировать в них разработанную библиотеку
    \begin{itemize}
        \item Для triton все реализовано в соответсвии с планом.
        \item Для libdft часть работ была проделана А. Харченко, из-за PinCRT было принято решение реализовать снятие интересующей информации без помощи \emph{instrumentation-lib}. Использовать утилиту на python на основе \emph{instrumentation-lib} для парсинга.
        \item Для taintgrind все реализовано в соответсвии с планом.
        \item Для moflow работа проделана Шамилем. (Но т.к. moflow в итоге был выбран, его изучение было проведено позднее)
    \end{itemize}
\end{frame}

\begin{frame}{Решаемые подзадачи | Сравнение инструментов }
    \textbf{Задача.} Сравнить работу инструментов на тестовых примерах из набора LAVA.\\
    \textbf{Результаты.} 
    \begin{itemize}
        \item Были разработаны скрипты для
        \begin{itemize}
          \item параллельного запуска инструмента на тестовых примерах.
          \item генерирование таблицы с результатами метрик.
        \end{itemize}
        \item на сервере ibis были проведены запуски инструментов и снятиы интересующие результаты.
    \end{itemize}
\end{frame}

\begin{frame}[fragile]{Решаемые подзадачи | Генерирование тестовых программ для оценки качества taint}
    \textbf{Задача.} Разработать генератор C программ, состоящих из последовательности вложенных if выражений над элеменатими последовательности де Брёйна.\\
    \textbf{Результаты: } Успешно генерируются программы вида:
    \begin{lstlisting}
int main(int argc, char** argv)
{
 ... // this is one-byte examle, 4-bytes generates as well 
 if (( data[ 0 ] | 110 ) < 143) {
  data[1] ^= 56;
  counter++;
  if (( data[ 1 ] ^ 192 ) != 50) {
   data[2] ^= 175;
   counter++;
  }
 }
 printf("%d", counter);
}
\end{lstlisting}
Все выражения внутри if оказываются истины, так как в противном случае условие заменяется на обратное в процессе генерации.
\end{frame}

\begin{frame}{Решаемые подзадачи | Доработки moflow}
    \textbf{Задача.} Улучшить механизм распространение пометок в Moflow.\\
    \textbf{Проблемы: }
    \begin{itemize}
        \item Нет поддержки нескольких тегов для области памяти/регистра.
        Специальный тег \emph{MIXED\_TAINT} для случая нескольких тегов. \textbf{(Решено)}
        \item Отсутсвие учета семантики инструкции при распространении пометок
        \item Поддерживаются только старшие регистры. \textbf{(В работе, планируется закончить до конца следующей недели)}
    \end{itemize}
\end{frame}

\appendix

%\begin{frame}[standout] \vfill Thanks for attention \vfill \end{frame}
\end{document}
