\documentclass[10pt]{beamer}

\usepackage[utf8]{inputenc}
\usepackage[english,russian]{babel}

\usetheme[
  sectionpage=simple,
  numbering=fraction
]{metropolis}
\usepackage{appendixnumberbeamer}

\usepackage{booktabs}
\usepackage[scale=2]{ccicons}

\usepackage{pgfplots}
\usepgfplotslibrary{dateplot}

\usepackage{xspace}
\newcommand{\themename}{\textbf{\textsc{metropolis}}\xspace}

\usepackage{todonotes}
\usepackage{ifthen}%
\providecommand\enabletodos{true}%
\ifthenelse{ \equal{\enabletodos}{true} }{%
  \presetkeys{todonotes}{inline}{}%
}{%
  \presetkeys{todonotes}{disable}{}%
}%

\setbeamertemplate{caption}{\raggedright\insertcaption\par}

\AtBeginSection[]{}

\setbeamertemplate{section in toc}{%
  \alert{$\bullet$}~\inserttocsection}
\setbeamercolor{subsection in toc}{bg=white,fg=structure}
\setbeamertemplate{subsection in toc}{%
  \hspace{1.2em}{\alert{\rule[0.3ex]{3pt}{3pt}}~\inserttocsubsection\par}}

\usepackage{hyperref}
\hypersetup{unicode=true}

\usepackage{tikz}
\usepackage{forest}
\usetikzlibrary{arrows,positioning}
\tikzset{
    % Define standard arrow tip
    >=latex,
    % Define arrow style
    ptr/.style={->, thick},
}
\usepackage{drawstack}
\usepackage{adjustbox}
\protected\def\psverb#1{\def\innerpsverb##1#1{\texttt{##1}}\innerpsverb}
\usepackage{listings}

\usepackage{makecell}

\usepackage[noline, noend]{algorithm2e} % For algorithms
\SetAlFnt{\footnotesize}

\usepackage{float}

\usepackage{changepage}
% \usepackage{pscyr} % Нормальные шрифты
 
% \usepackage{algorithm}
% \usepackage{algpseudocode}

\lstdefinestyle{C}{
  language=C,
  % numbers=left,
  stepnumber=1,
  % numbersep=10pt,
  % tabsize=4,
  showspaces=false,
  showstringspaces=false
}
\lstset{basicstyle=\tiny,style=C}


\metroset{block=transparent}
\definecolor{Blue}{HTML}{2375a8}
\setbeamercolor{frametitle}{bg=Blue}
\setbeamercolor{palette primary}{bg=Blue}


\definecolor{Blue2}{HTML}{9ccff0}
\setbeamercolor{block title}{bg=Blue2}

\title{Магистерская диссертация на тему}
\subtitle{Исследование и разработка методов динамического анализа для определения входных данных влияющих на выполнение условных переходов}
\author{Дьячков Л.А.\\[10mm]{\small Руководитель: к.ф-м.н, с.н.с. Курмангалеев Ш.Ф.}}
% \author{}
\institute{ИСП РАН}
\date{5 Апрeля 2019}
\titlegraphic{\hfill\includegraphics[height=0.5cm]{logo_isp_ru.png}}

\begin{document}

\maketitle

% {%
% \setbeamertemplate{frame footer}{
% Падарян В.А., Соловьев М.А., Кононов А.И. "Моделирование операционной семантики
% машинных инструкций." Труды Института системного программирования РАН, том 19,
% стр. 165186. 2010.
% }
\begin{frame}{Введение}
  \begin{block}{Фаззинг}
  \textbf{Фаззинг-тестирование} -- активно развивающийся метод для поиска ошибок в программном обеспечении %\cite{DBLP:journals/corr/abs-1808-09700}.
  \end{block}
  \pause
  \begin{block}{Проблема}
  Может быть затруднено нахождение входных данных, позволяющих ``пройти'' условный переход
  \end{block}
  \pause
  \begin{block}{Предлагаемое решение}
    Использовать методы динамического анализа для определения какие байты из входного файла влияют на конкретный условный переход
  \end{block}
\end{frame}

\begin{frame}{Постановка задачи}

  \begin{block}{Цель работы}
    \begin{itemize}
    \item Разработка метода динамического анализа, позволяющего определять байты входного файла, влияющего на выполнение инструкции условного перехода.
    \item Программная реализация метода, работающая на операционной системе Linux с архитектурой процессора x86-64
    \end{itemize}
  \end{block}
  \pause

  \begin{block}{Подзадачи}
    \begin{itemize}
      \item Изучение существующих технологий динамического символьного выполнение и динамического анализа потока данных.
      \item Сравнение технологий на тестовом наборе
      \item Обзор возможных подходов, реализация прототипов, разработка метода
      \item Программная реализация на основе выбранной технологии
    \end{itemize}
  \end{block}

    % \textbf{Задача} Для сравнения инструментов использовать существующию инфраструктуру Google OSS-fuzz \\
    % \textbf{Результаты}
    % \begin{itemize}
    %   \item Изучена инфраструктура google os fuzz.
    %   \item Получено представление о работе afl и libfuzz (в меньшей степени) c точки зрения пользователя
    %   \item На jenkins заведен job, запускающий afl фаззер при помощи oss fuzz
    %   \item Понимание как адаптировать имеющуюся инфраструктуру для оценки инструментов taint анализа не получено
    % \end{itemize}
\end{frame}
%}%

\begin{frame}{Динамическое символьное выполнение}
  \begin{block}{Определение}
    Метод динамического анализа, заключающийся в том в том, что во время выполнения программы некоторым конкретным значениям ставятся в соответствие символьные переменные. Затем, для каждой выполняемой инструкции, генерируются формулы для SMT решателя.
  \end{block}
  \begin{block}{online символьное выполнение}
    Модуль символьного выполнения используется для того, чтобы генерировать конкретные данные для посещения новых узлов cfg программы.
  \end{block}
  \begin{block}{Offline символьное выполнение}
    Модуль символьного выполнения используется для анализа конкретной трассы выполнения.\pause \checkmark
  \end{block}
\end{frame}


\begin{frame}{Динамический анализ помеченных данных}
  \begin{block}{Определение}
    Динамический анализ помеченных данных (Dynamic Taint Analysis), также известный как динамический анализ потока данных (Dynamic Flow tracking) -- это техника анализа програм, позволяющая определить какие состояния программы зависят от входных данных.
  \end{block}
  \begin{block}{Принцип работы}
    \begin{itemize}
        \item {\em Определение источников помеченных данных}. Обычно метками снабжаются данные, получаемые из недоверенного источника (файл, stdin, сеть).
        \item {\em Распространение пометок (Taint propogation)}. Для каждой инструкции необходимо принять решение как распротраняются пометки в зависимости от её операндов и факта их помеченности
        % \begin{itemize}
        %     \item Следует ли отслеживать помеченность побайтово или побитого? Если \textint{eax} помечен, то после команды \textint{or eax, 0x746567bc} контролируются уже не все биты. Однако, в большинстве случаев отслеживание каждого бита может быть слишком дорогой операцией.
        %     \item Следует ли помечать адрес памяти, на который указывает помеченная переменная?
        %     \item Если условный переход зависит от помеченных данных, следует ли считать что последующие инструкции тоже от них зависят?
        %     \item Как хранить информацию о помеченных адресах в памяти?
        %     \item Cледует ли различать пометки, полученные из разных источников?
        % \end{itemize}
        \item {\em Применение политик безопасности}. Например отслеживание попадания помеченных данных в аргументы ``опасных'' функций, или факта помеченности счетчика инструкций.
    \end{itemize}
  \end{block}
\end{frame}

% \begin{frame}{Проблема распространения пометок}
%     \begin{itemize}
%         \item Следует ли отслеживать помеченность побайтово или побитого? Если \textit{eax} помечен, то после команды \textit{or eax, 0x746567bc} контролируются уже не все биты. 
%         \item Следует ли помечать адрес памяти, на который указывает помеченная переменная?
%         \item Если условный переход зависит от помеченных данных, следует ли считать что последующие инструкции тоже от них зависят?
%         \item Как хранить информацию о помеченных адресах в памяти?
%         \item Cледует ли различать пометки, полученные из разных источников?
%     \end{itemize}
% \end{frame}

\begin{frame}{Исследованные технологии}
  \begin{block}{Динамическое символьное выполнение}
    \begin{itemize}
      \item Triton
      \item Angr
      \item Manticore
    \end{itemize}
  \end{block}
    \begin{block}{Динамический анализ помеченных данных}
    \begin{itemize}
      \item Triton 
      \item Taintgrind
      \item libdft64
      \item moflow (bap gentrace)
    \end{itemize}
  \end{block}
\end{frame}

\begin{frame}{Triton}
    Фреймворк, поддерживающий архитектуры x86 и x86-64, содержит модули DSE (offline) и анализа помеченных данных, использует \emph{Intel Pin} для динамической бинарной инструментации.
    \begin{block}{Достоинства}
        \begin{itemize}
        \item Поддерживает offline режим динамического символьного выполнения (Нет необходимости предварительно снимать трасу, при помощи pin это делается ``на лету'')
        \item Поддерживает \textbf{"ONLY\_TAINTED"} режим, позволяющий генерировать smt формулы только для помеченных инструкций
        \end{itemize}
    \end{block}

      \begin{block}{Недостатки}
          \begin{itemize}
      \item Работает крайне медленно (примерно на два порядка медленее других инструментов на базе pin)
      \item Нет поддержки символьных файлов/пометки данных на основе системных вызовов
      \item Грубый модуль пометки данных, низкая гранулярность over-approximation
      \end{itemize}
    \end{block}

\end{frame}

% \begin{frame}{Triton | Достоинства}
%     \begin{itemize}
%         \item Поддерживает offline режим динамического символьного выполнения (Нет необходимости предварительно снимать трасу, при помощи pin это делается ``на лету'')
%         \item Поддерживает \textbf{"ONLY\_TAINTED"}, позволяющий генерировать smt формулы только для
%         \item Богатое и документированное api c множеством примеров.
%         \item Поддерживает интерфейс для использования средств динамической бинарной инструментации, отличных от pin
%     \end{itemize}
% \end{frame}

% \begin{frame}[fragile]{Triton | Недостатки}
%       \begin{itemize}
%       \item 
%       \item Генерирует громоздкие SMT формулы для SSE инструкций. Предикат пути для примера ниже не решается в течение месяца.
%       \begin{verbatim}
%         for (int i = 0; i < arg_length; i++) {
%             buff[i] = (++argv[1][i]);
%         }
%         if (!strncmp(buff, "/home/", 6))
%         \end{verbatim}
%       \item Нет поддержки символьных файлов/пометки данных на основе системных вызовов.
%       \item Cодержит ошибки в реализации обработчиков нескольких инструкций
%       \begin{itemize}
%               \item Ошибка распространения пометов для инструкции \textbf{pcmpeqb} https://github.com/JonathanSalwan/Triton/issues/730
%               \item Неверный предикат пути для инструкции \textbf{repe cmpsb}
%           \end{itemize}
%       \end{itemize}
% \end{frame}

\begin{frame}{Angr}
    Платфмормонезависимый фреймворк динамического символьного выполнения, использующий трансляцию инструкций в VEX с последующей эмуляцией.
    \begin{block}{Достоинства}
      \begin{itemize}
        \item Поддержка множества архитектур
        \item Хороший онлайн движок, отлично работающий поиск состояний на небольших примерах.
        \item Есть множество полезных примитивов (таких как символьные файлы) из коробки
      \end{itemize}
    \end{block}
        \begin{block}{Недостатки}
          \begin{itemize}
      \item Нет оффлайн режима (плагины, которые должны его поддерживать не работают)
      \item На настоящих (например binutils) программах онлайн поиск не может дойти до интересных состояний 
      \end{itemize}
    \end{block}
\end{frame}


\begin{frame}{Manticore}
    Аналог Angr, поддерживающий также смарт-контракты. использует эмуляцию инструкций.
    \begin{block}{Достоинства}
      \begin{itemize}
        \item Поддерживает offline режим % на самом деле нет
      \end{itemize}
    \end{block}
        \begin{block}{Недостатки}
          \begin{itemize}
      \item Медленно работает (около 50 секунд эмуляции для того, чтобы дойти до main в программе с glibc)
      \item Нет поддержки SSE инструкций
      \item Нет поддержки некоторых системных вызовов при эмуляции
      \end{itemize}
    \end{block}
\end{frame}

\begin{frame}{Taintgrind}
    Плагин для valgrind, реализующий динамической анализ потока данных.
    \begin{block}{Достоинства}
      \begin{itemize}
        \item Поддержка множества архитектур (все что поддерживает valgrind)
        \item Возможность помечать данные из конкретных файлов
        \item Возможность статической инструментации исходного кода для пометки данных
      \end{itemize}
    \end{block}
        \begin{block}{Недостатки}
          \begin{itemize}
      \item Низкий уровень гранулярности пометок
      \end{itemize}
    \end{block}
\end{frame}

\begin{frame}{libdft64}
    Инструмент для динамического анализа помеченных данных, работающий на основе \emph{Intel Pin}. Реализация libdft (поддерживает только x86) из проекта vuzzer64 работающая с 64 разрядными исполняемыми файлами. 
    \begin{block}{Достоинства}
      \begin{itemize}
        \item Возможность помечать данные из конкретных файлов
        \item Гранулярность меток на уровне байта
      \end{itemize}
    \end{block}
        \begin{block}{Недостатки}
          \begin{itemize}
      \item Игнорируется регистр флагов
      \item Поддерживаются не все инструкции
      \end{itemize}
    \end{block}
\end{frame}


\begin{frame}{moflow (gentrace)}
    Инструмент для динамического анализа помеченных данных, работающий на основе \emph{Intel Pin}.
    Часть фреймворка moflow, изначально являющийся частью bap.
    \begin{block}{Достоинства}
      \begin{itemize}
        \item Возможность помечать данные из конкретных файлов
        \item Гранулярность меток на уровне байта
      \end{itemize}
    \end{block}
        \begin{block}{Недостатки}
          \begin{itemize}
      \item Отсутсвие учета семантики инструкций в механизме распространения пометок
      \item Использование специальной метки \emph{MIXED\_TAINT} для случая, когда адрес зависит от нескольких пометок.
      \item Механизм распространения пометок работает на уровне старших регистров
      \end{itemize}
    \end{block}
\end{frame}

\begin{frame}{Метод на основе символьного выполнения}

Каждому байту входного файла ставится в соответствие символьная переменная.
Для каждого предиката пути выполняется следующий алгоритм

\begin{algorithm}[H]
\SetAlgoLined
\KwIn{Предикат пути, представленный в виде AST для SMT формулы}
\KwOut{Список символьных переменных}
\SetKwFunction{getleafs}{\textbf{GetLeafs}}
\SetKwFunction{isleaf}{\textbf{IsLeaf}}
% \Indm\nonl\printlcs{$V$}\\
% \KwResult{Write here the result
\SetKwProg{Fn}{Function}{:}{}
\Fn{\getleafs{$V$}}{
$Found \gets \emptyset$\;
  \For{$child \in V$} {
    \If{$\isleaf ( child ) $} {
        $Found \gets Found \cup \{ child \}$\;
    } \Else {
      $Found \gets Found \cup \getleafs{child}$\;
    }
  }
  \Return{$Found$}\;
}
  \caption{Метод на основе символьного выполнения}
\end{algorithm}
Поскольку переменные взаимнооднозначно соответствуют адресам -- задача решена.
\end{frame}


\begin{frame}{Метод на основе символьного выполнения}

\begin{block}{Достоинства}
  \begin{itemize}
    \item Простой и естественный алгоритм
    \item Высокая точность работы
  \end{itemize}
\end{block}

\begin{block}{Недостатки}
  \begin{itemize}
    \item Низкая производительность (Построение формул на каждую инструкцию)
    \item Ни один из расмотренных DSE инструментов не позволяет реализовать метод, работающий на несинтетических примерах без ощутимых доработок самого инструмента
  \end{itemize}
\end{block}

\end{frame}

\begin{frame}{Метод на основе символьного выполнения}

\begin{block}{Достоинства}
  \begin{itemize}
    \item Простой и естественный алгоритм
    \item Высокая точность работы
  \end{itemize}
\end{block}
\pause
  \begin{block}{Недостатки}
    \begin{itemize}
      \item Низкая производительность (Построение формул на каждую инструкцию)
      \item Ни один из расмотренных DSE инструментов не позволяет реализовать метод, работающий на несинтетических примерах без ощутимых доработок самого инструмента
    \end{itemize}
  \end{block}
\end{frame}


\begin{frame}{Метод на основе анализа помеченных данных}

    \begin{itemize}
      \item Каждому байту входного файла ставится в соответсвие метка (тег).
      \item Для всех последующих инструкций в трассе выполнение выполняется алгоритм распространения пометок
      \item Для каждой инструкции условного извлекаются теги, которыми помечен регистр флагов. Байты, соответствуюшие этим тегам -- искомые.
    \end{itemize}
\end{frame}

\begin{frame}{Метод на основе анализа помеченных данных}

\begin{block}{Достоинства}
  \begin{itemize}
    \item Высокая скорость работы
    \item Простая реализация
  \end{itemize}
\end{block}
\pause
  \begin{block}{Недостатки}
    \begin{itemize}
      \item Точность ниже чем у символьного выполнения.
    \end{itemize}
  \end{block}
\end{frame}

\begin{frame}{Реaлизация}
Метод на основе динамического символьного выполнения был реализован на фреймворке \textbf{Angr}. Для промышленного применения не подходит по озвученным выше причинам.
\\
\pause
Для реализации метода на основе анализа помеченных данных был выбран Moflow.

\end{frame}


\begin{frame}[fragile]{Как работает Moflow}

    \scalebox{0.3}{\includegraphics{../../img/source_tainting.png}}

\end{frame}


\begin{frame}[fragile]{Доработки Moflow}
    % \hfill
    % {\scriptsize\input{mydrawio.pdf_tex}}
    \textbf{RCX} помечен $8$ тегами. Moflow бы пометил бы
    \textbf{RCX} как \textbf{MIXED\_TAINT}, для реализации множественного помечивания нужна  структура \textbf{Множество пометок}.

    \scalebox{0.3}{\includegraphics{../../img/propagation1.png}}

\end{frame}

\begin{frame}[fragile]{ Доработки Moflow}
    \textbf{0x5678fff0} и \textbf{0x5678fff1} помечаются $8$ тэгами. Правильно было бы представить \textbf{RCX} как 8 отдельных байт, которые могут быть помечены независимо, тогда \textbf{0x5678fff0} и \textbf{0x5678fff1} будут помечены $2$ различными тегами.
    \scalebox{0.25}{\includegraphics{../../img/propagation2.png}}

\end{frame}


\begin{frame}{Доработки Moflow | Множество пометок}
    
    \textbf{Множество пометок} -- структура для поддержки множественных пометок, от неё требуется поддержка следующих операций
    \begin{itemize}
        \item Добавление \textbf{пометки в множество}.
        \item Объединение с другим \textbf{Множеством пометок}
        \item Вывод содержимого \textbf{Множества пометок}
     \end{itemize}
     \pause
     Было разработано несколько подходов к реализации структуры.

\end{frame}

\begin{frame}{Доработки Moflow | Множество пометок}

    \textbf{Время работы в секундах для различных реализаций}\\
    \scalebox{0.75}{
    \begin{tabular}[]{@{}llllllll@{}}
    \toprule
    & vanilla & vector & bitset6000 & roaring & bitset64 tree & bitset256
    tree & bitset512 tree\tabularnewline
    \midrule
    % \endhead
    cmark & 3.5s & 4.6s & 3.7s & 4.2s & 3.8s & 3.7s & 3.7s\tabularnewline
    file & 20.8s & 47s & 60s & 70s & 45.5s & 46.3s & 47.5s\tabularnewline
    libjpeg & 14.5s & 1762s & 49s & 378s & 307s & 108s & 80s\tabularnewline
    libyaml & 16.5s & 22.5s & 25s & 26s & 23.5s & 23.5s & 23.5s\tabularnewline
    \bottomrule
\end{tabular}}
\end{frame}


\begin{frame}{Доработки Moflow | Множество пометок | Комментарии }

\begin{itemize}
  \item Поскольку не менее 70\% операций объединения проводится над пустыми множествами и множествами из $1$ элемента, во всех случаях множество пометок было реализовано как пара из \textbf{uint32\_t} и указателя на более сложную структуру (который в этих 70\% оказывается нулевым).
  \item Обычные \textbf{std::set} и \textbf{std::unordered\_set} оказались заметно медленее приведенных реализаций, и в таблицу не сключены.
  \item Roaring (https://github.com/RoaringBitmap/CRoaring) -- библиотека сжатых битовых векторов.
\item Bitset -- реализация на основе \textbf{std::bitset}, $6000$ битов достаточно для всех примеров из тествого набора. Размер битового множества должен быть известенна этапе компиляции.
\item \textbf{bitset64 tree} и \textbf{bitset256 tree} реализации, в которых \textbf{Множество пометок} представлено как a \textbf{std::map} с целочисленными ключами и \textbf{std::bitset} размеров $64$ и $256$ в качестве значений.
\end{itemize}
\end{frame}


\appendix
%\begin{frame}[standout] \vfill Thanks for attention \vfill \end{frame}
\end{document}
